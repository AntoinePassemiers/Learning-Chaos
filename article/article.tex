\documentclass[letterpaper]{article}
\usepackage{natbib,alifexi}
\usepackage{color}

\newcommand\todo[1]{\textcolor{red}{TODO: #1}}

\title{Human-level control through deep reinforcement learning}
\author{C\'{e}dric Simar, Antoine Passemiers, Stanislas Gueniffey \and Robin Petit \\
\mbox{}\\
Universit\'{e} Libre de Bruxelles \\
apassemi@ulb.ac.be}


\begin{document}
\maketitle

\begin{abstract}

  Reinforcement learning is a subarea of machine learning inspired by neuroscience, where learners have to select actions with
  regards to the environement state and receive a reward accordingly. Each agent's objective is to maximize its cumulative reward across its whole lifetime,
  subdivided into episodes. In the traditional framework of reinforcement learning and more specifically in Q-learning,
  researchers usually deal with a small discrete space to represent the values of Q. This is no longer sufficient if one desires to describe the environment state
  as an image of raw pixels. In consequence, we introduce deep Q-learning methods, which have been found to be really effective in mapping raw pixels
  to abstract high-level features. In deep Q-learning, the estimation of Q-values for a given action are produced using deep learning neural networks,
  which consist in stacks of many neural layers. These techniques have revealed themselves to be able to beat human experts at playing Atari games 
  by learning only from visual features. Despite the current progress, learning to take complex decisions on the basis of high-dimensional visual data remains on
  ongoing challenge.

\end{abstract}

\section{Introduction}

  \todo{pour Robin}

\section{Methods}

  \todo{pour Robin} \citep{Mnih2015}

\section{Results}

  \todo{Plots avec les erreurs d'apprentissage (pas de cross-val)}

  \todo{Magnifique plot T-SNE avec des couleurs et des gifs et des trucs qui font beep et qui font boop} \citep{wattenberg2016how}

\section{Discussion}

  \todo{Amélioration: Double dueling DQN} \citep{DBLP:journals/corr/WangFL15}

\section{Acknowledgements}

  \todo{Remercier Lenaerts, Nowe, Google, OpenAI, Markov}

\footnotesize
\bibliographystyle{apalike}
\bibliography{article}


\end{document}
